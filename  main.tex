\documentclass{article}
\usepackage[utf8]{inputenc} % Quan trọng cho tiếng Việt
\usepackage[T5]{fontenc} % Để hỗ trợ tiếng Việt nếu cần
\usepackage{hyperref} % Tốt cho các liên kết trong HTML
\usepackage{graphicx} % Để nhúng ảnh nếu có
\usepackage{float} % Để kiểm soát vị trí ảnh

\title{Tài liệu Thử nghiệm LaTeX sang PDF và HTML}
\author{Sangnhc}
\date{\today}

\begin{document}
\maketitle

\section{Giới thiệu}
Đây là một tài liệu LaTeX đơn giản để minh họa việc biên dịch sang PDF và chuyển đổi sang HTML bằng \texttt{make4ht}.

\subsection{Mục con: Tính năng}
Nội dung của mục con. Chúng ta có thể có \textbf{văn bản in đậm} và \textit{văn bản in nghiêng}.
\href{https://github.com}{Click vào đây để đến GitHub}.

\begin{figure}[H]
    \centering
    % Giả sử bạn có một file ảnh `example.png` trong repo
    % Để đơn giản, bạn có thể bỏ phần ảnh này nếu không có file ảnh
    % \includegraphics[width=0.5\textwidth]{example.png}
    \caption{Ví dụ Hình ảnh (nếu có)}
    \label{fig:example}
\end{figure}

\section{Kết luận}
Hy vọng quy trình này hoạt động tốt và bạn có thể xem tài liệu của mình dưới dạng PDF và HTML!
\end{document}